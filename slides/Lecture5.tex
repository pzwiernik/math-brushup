\documentclass[11pt,aspectratio=169]{beamer}
\input{./defs.tex}

\title[Calculus and Linear Algebra]{Lecture 5: Calculus and Linear Algebra}
\author[Piotr Zwiernik, Barcelona School of Economics]{Piotr Zwiernik\\[0.5ex]Mathematics Brush-up}
\titlegraphic{\includegraphics[width=1.5in]{img/bse.png}}
\date{}

\begin{document}

% ---------------- Title ----------------
\begin{frame}
  \titlepage
\end{frame}

% ---------------- Slide 1 ----------------
\begin{frame}{Motivation: why separable ODEs?}
\begin{small}
Ordinary differential equations (ODEs) appear whenever we model how a quantity changes over time or another variable.  

\medskip
A simple but very important class: \textcolor{SeaGreen}{separable ODEs}, where variables can be separated:
\[
\frac{dy}{dx} = g(x)\,h(y).
\]

\begin{itemize}
  \item Widely used in economics, biology, and physics for growth/decay models.
  \item Method: rearrange into \(\frac{dy}{h(y)} = g(x)\,dx\), then integrate both sides.
  \item Today: a very small illustrative example.
\end{itemize}
\end{small}
\end{frame}

% ---------------- Slide 2 ----------------
\begin{frame}{A toy separable ODE}
\begin{small}
Consider
\[
v'(x)\;=\;\frac{{\rm d}v}{{\rm d}x}\;=\;-\frac{1}{x}\,v(x).
\]

\medskip
\textcolor{SeaGreen}{Step 1: Separate variables.}
\[
\frac{dv}{v} = -\frac{dx}{x}.
\]

\textcolor{SeaGreen}{Step 2: Integrate both sides.}
\[
\int \frac{dv}{v} = -\int \frac{dx}{x}
\quad\Rightarrow\quad
\ln|v(x)|=-\ln|x|+C.
\]

Hence
\[
v(x)=C_1\,x^{-1}, \quad x\neq 0.
\]
\end{small}
\end{frame}

% ---------------- Slide 3 ----------------
\begin{frame}{Recovering $u$ from $v=u'$}
\begin{small}
Suppose \(v(x)=u'(x)\). Then
\[
u'(x)=C_1\,x^{-1}.
\]

\textcolor{SeaGreen}{Integrate again:}
\[
u(x)=C_1\ln|x|+C_2.
\]

\medskip
\textbf{Domain note:} Because of $\ln|x|$, we must restrict to $x>0$ or $x<0$ separately.
\end{small}
\end{frame}

% ---------------- Slide 4 ----------------
\begin{frame}{Example with initial conditions}
\begin{small}
Suppose $u(1)=0$ and $u'(1)=2$.

\begin{align*}
u'(1)=2 &\;\Rightarrow\; C_1=2, \\
u(1)=0 &\;\Rightarrow\; 0=2\ln 1+C_2 \;\Rightarrow\; C_2=0.
\end{align*}

Therefore
\[
u(x)=2\ln x \quad \text{(valid on $x>0$)}.
\]
\end{small}
\end{frame}

% ---------------- Slide 5 ----------------
\begin{frame}{Key takeaways}
\begin{small}
\begin{itemize}
  \item A \textcolor{SeaGreen}{separable ODE} can be solved by separating $y$-terms and $x$-terms, then integrating.
  \item Constants of integration appear at each step and are fixed by initial conditions.
  \item Always check the domain: functions like $\ln|x|$ force $x>0$ or $x<0$.
  \item Even this tiny example illustrates the general workflow.
\end{itemize}
\end{small}
\end{frame}
\end{document}